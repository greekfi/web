\documentclass[%
 reprint,
%superscriptaddress,
%groupedaddress,
%unsortedaddress,
%runinaddress,
%frontmatterverbose, 
%preprint,
%preprintnumbers,
%nofootinbib,
%nobibnotes,
%bibnotes,
 amsmath,amssymb,
 aps,
 prl,
%pra,
%prb,
%rmp,
%prstab,
%prstper,
floatfix,
]{revtex4-2}

\usepackage{graphicx}% Include figure files
\usepackage{subcaption}
% \usepackage{subfigure}

\usepackage{dcolumn}% Align table columns on decimal point
\usepackage{bm}% bold math
%\usepackage{hyperref}% add hypertext capabilities
%\usepackage[mathlines]{lineno}% Enable numbering of text and display math
%\linenumbers\relax % Commence numbering lines

%\usepackage[showframe,%Uncomment any one of the following lines to test 
%%scale=0.7, marginratio={1:1, 2:3}, ignoreall,% default settings
%%text={7in,10in},centering,
%%margin=1.5in,
%%total={6.5in,8.75in}, top=1.2in, left=0.9in, includefoot,
%%height=10in,a5paper,hmargin={3cm,0.8in},
%]{geometry}

% Fix excessive list spacing in REVTeX
\usepackage{paralist}
\setdefaultleftmargin{2em}{}{}{}{}{}

\newcommand{\greekfi}{Greek.fi }


\begin{document}

\title{\greekfi Options Protocol}

\author{Mahmoud Lababidi}
% \email{ml@greek.fi}
\affiliation{%
ml@greek.fi
}

\date{June 26, 2025}

\begin{abstract}
\greekfi is a protocol that enables decentralized American-style exercisable (American) options
that are fully collateralized.
It is designed for composability, exercisability, and universal use
on any EVM-compatible chain and allows any ERC20 token to be used as
collateral and consideration, including wBTC, wETH, stETH, USDC, and USDT.

This new level of composability enables new options ecosystem that allows and goes beyond 
traditional financial options strategies. 
Greek achieves capital efficiency through decentralized lending.

\end{abstract}

\maketitle

\section{Introduction}

% \subsection{Background}
Crypto Derivatives have exploded; across 24-hour averages, 
over \$6B of trading volume occurs in perpetual markets and Deribit was sold for \$3B. 
This ramp-up to catch the traditional finance options market (\$2.7T USD in daily notional)
points us that it's up only.

The DeFi space is held back from options growth: early exercisable (American) options do not exist in a liquid, decentralized, tokenized manner which will allow options to be as common as are yield generating tokens. 
Early exercise is an added benefit in options because it gives in-the-money option holders to capitalize on their investment.

To solve this, we introduce the \greekfi protocol, 
the first fully decentralized, collateralized, tokenized, 
exercisable, expirable options protocol in the Ethereum ecosystem.
The protocol achieves the following, which fulfill the requirements for American options:
\begin{itemize}
  \setlength{\itemsep}{0pt}
  \setlength{\parskip}{0pt}
  \item ERC20 compatible - allowing full composability
  \item Exercisable and collateralized protocol so that every option can be exercised to swap for the collateral
  \item The option writer can redeem consideration (Cash) prior to expiration and collateral post expiration
  \item The option writer can redeem collateral prior to expiration if they also own the option
  \item Dual token system (OPTION and REDEMPTION) similar to yield platforms
  \item Oracle-free and lending capable for capital efficiency
\end{itemize}

This protocol will unlock a trove of possibilities to DeFi, for example:

\begin{itemize}
  \setlength{\itemsep}{0pt}
  \setlength{\parskip}{0pt}
  \item Hedge short-term and long-term risk 
  \item Use leverage without getting margin called
  \item Earn yield against volatility
  \item Create unique underlying swap pairs \\ (wBTC/wETH, ETH/AAVE, USDe/USDC)
  \item Transfer options across ecosystems and chains
  \item Trade options in both OTC and other markets
  \item Cash Settlement through flash-loans
\end{itemize}

\section{Protocol Overview}
\paragraph*{Protocol Summary -}
The protocol uses two coupled ERC20 tokens to represent an option position: an Option Token and a Redemption Token. As an American option, the Option Token can be exercised at any time before expiration and Redemption Tokens allow redemption after expiration. 

\subsection{Protocol Details}
Options are represented by ERC20 Option Tokens (OT) that allow a user to excercise at specific strike price of N units of consideration for a specific underlying token, before an expiration date (Consideration is typically a cash asset like USDC, and Collateral is typically an asset like wETH).
This OT is unique, and temporary as it has an expiration date, after which it is worthless. 

To write an Option, we mint an OT by collateralizing assets which also results in a Redemption Token (RT) with the OT. 
The RT, also an ERC20, represents the short option position and redeemability of the asset that was collateralized, which can happen after expiration.
Also, the reverse of minting, combining the OT with the RT, also redeems Collateral.
Both tokens can be swap and delivered through DEXs/AMMs and RFQs on-chain.

The creation of the contract pair (OT + RT) is done throught a factory contract, which allows for upgradability of the protocol. 
This also helps in managing all the contracts created as a source of truth.
Each OT+RT pair is represented by a tuple of the following parameters:

\begin{table}[ht]
\centering
\begin{tabular}{|p{2.5cm} |}
\hline
Collateral \\
\hline
Consideration \\
\hline
Strike \\
\hline
Expiration \\
\hline
\end{tabular}
\end{table}


Other than standard ERC20 functionality, the OT permits a user to \textbf{exercise} the option and
\textbf{redeem} the underlying collateral if and only if they hold the RT as well.

\begin{table}[ht]
\centering
\begin{tabular}{|p{3.5cm} p{4cm}|}
\hline
\textbf{Option Token} & \\
\hline
exercise() & Exercise option: \\ &  swap consideration \\ & for collateral \\
\hline
redeem() & Redeem collateral \\ & pre expiration \\ & while holding OPTION \\
\hline
\textbf{Redemption Token} &  \\
\hline
redeem() & Redeems collateral \\ & after expiration \\
\hline
redeemConsideration() & Redeems consideration \\ & if available \\ 
\hline
\end{tabular}
\caption{Available functions for Option and Redemption Tokens}
\label{tab:functions}
\end{table}


\subsection{Capital Efficiency}

To achieve capital efficiency, we need to use less collateral to achieve the same exposure. 
In traditional finance, the options clearing houses use margining to achieve this.
In DeFi, we can use lending protocols to achieve this.

Let's say we need 1 wETH as collateral to mint an option.
We borrow 0.7 wETH from a lending protocol, combined with 0.3 wETH of our own, we can mint the option.
In return for the loan, we return the RT back to the lender as collateral for the loan and
sell the OT to earn premium.

To retrieve our collateral, we need to repay the loan, which can be done by buying back the OT from the market.
The OT is combined with the RT to redeem the collateral, and we return the borrowed wETH to the lender.

Lastly, if the lender needs to margin call against the RT, they buy the coupled OT,
unlock the collateral and return the remainder back to the borrower.
The LTV limit is determined by the Strike and the spot price of the collateral.

\begin{figure}[ht]
  \centering
  \includegraphics[width=0.45\textwidth]{short_price.png}
  \caption{Price of the RT in the two regimes. The orange solid line shows the realistic price incorporating OT value, while the green dot-dashed line shows an extreme fair price for low-volume assets.}
  \label{fig:short_price}
\end{figure}


\section{Applications and Future Considerations}
\paragraph*{Trading}

The composability of the protocol allows for multiple trading venues for the options on-chain.
Traditional DEX AMMs, such as Uniswap, cannot handle the low volume and low liquidity and atypical ($xy=k$) pricing that options could have. 
The obvious solution is to use RFQs, such as 0x, Bebop, UniswapX etc. enabling market makers (MMs) to provide liquidity and pricing for the options.
OTs can be minted within the transfer() function to allow for seamless OTC trading

\paragraph*{Default Swaps}

The protocol is able to provide default swaps as insurance for risk management in debt markets. 
An Oracle or Third party can \textbf{unlock} the OT to allow exercise.

\paragraph*{Cash Settlement}
Through a Flash Loan (via a swap protocol, ie Uniswap), one can borrow Consideration, exercise, repay via collateral, and be settled in one transaction.


\paragraph*{Crosschain interoperability}
With protocols such as LayerZero, Hop Protocol, and others, crosschain interoperability is becoming more common.
This system can be designed to be interoperable across chains, with careful design as there are +3 additional moving parts: Collateral, Consideration, and Option/Redemption tokens across chains. 
We can upgrade the factory contract to support crosschain options.

\paragraph*{Option Drops instead of Token drops}
In TradFi, options are vested by startup employees and exercised for tax purposes.
A similar use case can be made here to provide option drops instead of token drops.

\paragraph*{Compound Options}
A compound option is an option on an option. Yes, this is a thing. It's basically a second derivative.
And yes, this protocol can handle it. 

\paragraph*{Covered Call Vaults}

Other protocols have created covered call vaults, such as Rysk Finance.
This approach can be easily implemented using the \greekfi protocol.
PropAMMs and RFQ market makers can price options for the vaults.
This strategy is nearly identical to ETF strategies (i.e., XYLD).
This simplifies the user experience and reduces the amount of work
required to earn yield. The same strategy can be applied to
covered put vaults as well.


\end{document}
